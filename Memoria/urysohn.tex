

\section{El Lema de Urysohn}

[Introducción] \redcode{no se que poner}

\begin{theorem}[\textbf{Lema de Urysohn}]
  Sea $(X, \mathcal{T})$ un espacio topológico. $X$ es un espacio normal si y solo si para cada par de conjuntos cerrados disjuntos $C$ y $D$ en $X$, existe una función $f : X \to [0, 1]$ de manera que $f(C) = \{0\}$ y $f(D) = \{1\}$.
\end{theorem}

Para la fomalización en Lean, pediremos que los cerrados $C$ y $D$ sean no vacíos. Obviamente, si uno de los dos es vacío, basta tomar la función continua $f(x) \equiv 1$, pero podemos descartar estos casos triviales.

\begin{lstlisting}
  lemma Urysohn {X : Type} {Y : Set ℝ} (T : TopologicalSpace X)
      [T' : TopologicalSpace ℝ] (hT' : T' = UsualTopology)
      {R : TopologicalSpace Y} {hY : Y = Set.Icc 0 1}
      {hR : R = TopoSubspace T' Y} :
      NormalSpace X ↔
        ∀ C1 : Set X, ∀ C2 : Set X, C1 ≠ ∅ → C2 ≠ ∅ →
        IsClosed C1 → IsClosed C2 → Disjoint C1 C2 →
        ∃ f : X → Y, Continuous f ∧
          f '' C1 = ({⟨0, by simp [hY]⟩} : Set Y) ∧
          f '' C2 = ({⟨1, by simp [hY]⟩} : Set Y) := by sorry
\end{lstlisting}

Veamos primero la demostración del recíproco, que es más sencilla.

\begin{proof}

  Supongamos que cualquier par de cerrados disjuntos de $X$ se pueden separar mediante una función continua y veamos que entonces $X$ es un espacio normal. Sean $C_1$ y $C_2$ cerrados disjuntos en $X$.

  \begin{lstlisting}
    · intro h
      rw [normal_space_def]
      intro C1 C2 hC1 hC2 hinter \end{lstlisting}

  Como en la definición de normal no pedimos que los cerrados sean no vacíos, tenemos que diferenciar estos casos. Sin embargo, estos casos son triviales porque basta tomar el conjunto vacío para recubrir el vacío y $X$ para recubrir el otro conjunto. En Lean hay que ser rigurosos con este paso, pero aquí lo obviaremos por simplicidad.

  Supongamos entonces que $C_1$ y $C_2$ son no vacíos. Por hipótesis, existe una función continua $f : X \to [0, 1]$ de forma que $f(C_1) = \{0\}$ y $f(C_2) = \{1\}$.

  \begin{lstlisting}
    obtain ⟨f, hf, hfC1, hfC2⟩ := h C1 C2 hC1nempty hC2nempty hC1 hC2 hinter \end{lstlisting}

  Consideremos entonces los conjuntos $U_1 = f^{-1}([0, \frac{1}{2}))$ y $U_2 = f^{-1}((\frac{1}{2}, 1])$. Queremos ver que son los abiertos que necesitamos de la definición de normal, es decir, que son abiertos en $X$, que $C_i \subseteq U_i$ y que son disjuntos.

  \begin{lstlisting}
    use f ⁻¹' ({y | (y : ℝ) ∈ Set.Ico 0 (1 / 2)})
    use f ⁻¹' ({y | (y : ℝ) ∈ Set.Ioc (1 / 2) 1}) \end{lstlisting}

  Para ver que $U_1$ es abierto, utilizamos que $f$ es continua. Basta ver que $[0, \frac{1}{2})$ es abierto en $[0, 1]$. Pero ya vimos que los intervalos de la forma $[0, b)$ son abiertos en $[0, 1]$, así que basta aplicar esta propiedad. Análogo para $U_2$.

  \begin{lstlisting}
    · apply hf -- aplicar def. de f continua
      apply ico_open_in_Icc01 -- `[0, b)` es abierto en `[0, 1]`
      · exact hY -- estamos en [0, 1]
      · exact hR -- estamos en la top. relativa
      · norm_num -- 0 < 1/2 < 1 \end{lstlisting}

  Para ver que $C_1 \subseteq U_1$, basta ver que $f(C_1) \subseteq [0, \frac{1}{2})$. Que es trivial. Análogo para $U_2$.

  \begin{lstlisting}
    · rw [← Set.image_subset_iff, hfC1] -- `{0} ⊆ [0, 1/2)` ?
      simp \end{lstlisting}

  Para ver que son disjuntos, vemos que $[0, \frac{1}{2})$ y $(\frac{1}{2}, 1]$ son disjuntos. Obviamente lo son, pero para Lean es un poco más complicado, así que procedemos por reducción al absurdo para poder simplificar las expresiones. Finalmente llegamos a que no existe un $x$ con $x < 1/2$ y $x > 1/2$.

  \begin{lstlisting}
    · apply Disjoint.preimage
      by_contra c
      rw [Set.disjoint_iff_inter_eq_empty, ← ne_eq, ← Set.nonempty_iff_ne_empty] at c
      obtain ⟨x, hxu, hxv⟩ := c
      simp at hxu hxv
      linarith \end{lstlisting}
\end{proof}


La otra implicación es mucho más compleja, especialmente en su formalización en Lean, como veremos a continuación.


\subsection{Esquema de la demostración}

Para demostrar esta implicación, dados dos cerrados disjuntos no vacíos de $X$, queremos construir una función continua que los separe. La construcción de esta función ha constituido la parte más costosa de este trabajo.

Vamos a ver un esquema de los pasos a seguir para construir tal función.

Supongamos que $X$ es un espacio normal, y sean $C_1$ y $C_2$ dos cerrados disjuntos no vacíos de $X$.

Consideremos $U_1 = X$ abierto. Consideremos el cerrado $C_1$ el abierto $C_2^c$ y aplicamos la caracterización de espacios normales (\ref{caracterizacion-normal}), obteniendo otro abierto $U_0$ de manera que
$$
C_1 \subseteq U_0 \subseteq \overline{U_0} \subseteq C_2^c \subseteq U_1 = X
$$
Podemos hacer lo mismo para $\overline{U_0}$ cerrado y $C_2^c$ abierto, obteniendo $U_{\frac{1}{2}}$ de forma que
$$
C_1 \subseteq U_0 \subseteq \overline{U_0} \subseteq U_{\frac{1}{2}} \subseteq \overline{U_{\frac{1}{2}}} \subseteq C_2^c \subseteq U_1
$$
En general, vamos a construir una sucesión de abiertos sobre $\mathbb{Q}\cap[0, 1]$, $\{U_p | p \in \mathbb{Q}\cap[0, 1]\}$, de manera que
\begin{equation}
  \forall p , q \in \mathbb{Q}, p < q \implies \overline{U_p} \subseteq U_q \tag{$\star$} \label{eq:star}
\end{equation}

Una vez tenemos esta sucesión, que en Lean será una función $G : \mathbb{Q}\cap[0, 1] \to \mathcal{P}(X)$, definimos otra función $F$ sobre $X$ que a cada $x \in X$ le hace corresponder el conjunto
$$
F(x) = \{p \in \mathbb{Q} ~|~ x \in G(p)\}
$$
Por último, tomaremos la función $f : X \to [0, 1]$ definida por
$$
f(x) = \textnormal{inf}~F(x)
$$
Esta función será la que utilicemos. Tendremos que demostrar que efectivamente toma valores en $[0, 1]$, que es continua y que separa nuestros conjuntos cerrados.

Sin embargo, una vez construidas estas funciones, este último paso es relativamente sencillo. La principal dificultad a la hora de formalizar esta demostración ha sido el uso de la inducción para construir la función $G$ y demostrar sus propiedades.

Como se puede apreciar en las primeras iteraciones de la construcción de cada $U_q$, esta sucesión se construye por inducción. Para poder hacer inducción sobre los racionales, nos basamos en que son numerables, y, en particular, en que $\mathbb{Q}\cap[0, 1]$ lo es. Vamos a encontrar una función $f : \mathbb{N} \to \mathbb{Q} \cap [0, 1]$ biyectiva (invertible), de manera que $f(0) = 1$ y $f(1) = 0$. Esto nos servirá para construir cada $U_q$.

Después, para demostrar que efectivamente se cumple la condición (\ref{eq:star}), necesitaremos utilizar inducción sobre dos variables. Para ello, vamos a demostrar que el orden lexicográfico de $(\mathbb{N} \times \mathbb{N})$, definido por $(n, m) < (n', m') \iff n<n' \lor (n=n' \land m<m')$, es una relación bien fundada, y por tanto admite inducción sobre pares de naturales.

\begin{lstlisting}
  Nota: no se como poner que lo más difícil ha sido que no se me han ocurrido estas cosas a la primera jajajaja
\end{lstlisting}

\subsection{Construcción de la sucesión de abiertos}

Sea $Q = \mathbb{Q}\cap[0, 1]$ Para construir la sucesión $\{U_p ~|~ p \in Q\}$, o, lo que es lo mismo, la función $G : Q \to \mathcal{P}(X)$, $G(p) = U_p$, vamos a proceder de la siguiente forma.

Como $Q$ es numerable, consideramos la numeración $Q = \{p_k ~|~ k \in \mathbb{N}\}$, y supongamos por simplicidad que $p_0 = 1$ y que $p_1 = 0$. Vamos a construir los abiertos con la condición (\ref{eq:star}) por inducción sobre $k$.

Tomamos entonces, como base de la inducción,

\begin{itemize}
  \item $U_1 = U_{p_0} = X$.
  \item $U_0 = U_{p_1}$ el abierto obtenido al aplicar la caracterización de espacios normales (\ref{caracterizacion-normal}) al cerrado $C_1$ y el abierto $C_2^c$.
\end{itemize}

Trivialmente se tiene que $\overline{U_0} \subseteq U_1 = X$ (con $0<1$), luego se satisface (\ref{eq:star}).

Ahora, para el caso inductivo, supongamos que hemos definido $U_{p_k}$ para cada $k = 0, 1, ..., n$ satisfaciendo (\ref{eq:star}), y ahora queremos construir $U_{p_{n+1}}$.

Notar que el conjunto $\{p_0, p_1, ..., p_n\} \subset Q$ no está necesariamente ordenado. Sin embargo, es un conjunto finito de números racionales, por tanto, podemos encontrar unos elementos $p_r$ y $p_s$ de manera que $p_r$ es el predecesor inmediato de $p_{n+1}$ y $p_s$ es el sucesor inmediato de $p_{n+1}$. Es decir, se tiene
$$
p_r < p_{n+1} < p_s
$$
y no existe $k\leq n$ de forma que $p_r < p_k < p_{n+1}$ ni $p_{n+1} < p_k < p_s$.

Puesto que $p_r < p_s$, por la hipótesis de inducción se tiene que ambos son abiertos y que $\overline{U_{p_r}} \subseteq U_{p_s}$ (\ref{eq:star}). Aplicamos la caracterización de normalidad para encontrar un nuevo abierto $U = U_{p_{n+1}}$ tal que
$$
\overline{U_{p_r}} \subseteq U \subseteq \overline{U} \subseteq U_{p_s}
$$
Con esto concluimos la inducción.

Veamos ahora como se traduce esto en Lean.

\subsubsection{Numerar los racionales}

Los racionales son numerables, es decir existe una biyección entre $\nat$ y $\rat$. En particular, necesitamos una función $f : \nat \to Q$ donde $Q = \rat \cap [0, 1]$, de forma que $f$ sea biyectiva, $f(0) = 1$ y $f(1) = 0$. Es decir, la función que nos lleva cada $k \in \nat$ a $p_k$.

\begin{lstlisting}
  lemma hf : ∃ f : ℕ → Q,
      (f.Bijective ∧
      f 0 = ⟨1, Q1⟩ ∧
      f 1 = ⟨0, Q0⟩) := by sorry
\end{lstlisting}

Para demostrar la existencia de tal función necesitamos una serie de resultados previos.

En primer lugar, la numerabilidad de los racionales ya está demostrada en Mathlib con el resultado \code{Rat.instDenumerable}. Para extraer una función biyectiva de este resultado, he escrito el siguiente lema:

\begin{lstlisting}
  lemma bijective_nat_rat : ∃ f : ℕ → ℚ, f.Bijective  := by
    have f := (Rat.instDenumerable.eqv).symm
    use f
    exact f.bijective
\end{lstlisting}

Evidentemente, por la independencia de las demostraciones de Lean, no podremos evaluar esta función de manera explícita. Pero tenemos la información que necesitamos de ella.

Ahora, quiero demostrar que existe una función biyectiva de $\nat$ en $Q$. Como ya tenemos una función biyectiva de $\nat$ en $\rat$, la idea es componerla con una biyección de $\rat$ en $Q$.

Para demostrar que esta biyección existe, basta demostrar que $\rat$ y $Q$ tienen el mismo cardinal (\code{Cardinal.eq}). Pero, de hecho, cualquier subconjunto de $\rat$ no finito tiene cardinal $\aleph_0$ (demostrado en \code{non\_finite\_rat\_set\_cardinal\_aleph0}). Basta demostrar que $Q$ no es finito.

\begin{lstlisting}
  lemma non_finite_rat_set_cardinal_aleph0 (A : Set ℚ) (hA : ¬ A.Finite) :
      Cardinal.mk ↑A = Cardinal.aleph0 := by sorry
\end{lstlisting}

Por último, cualquier permutación de dos valores de una función preserva la biyectividad (demostrado en \code{permute\_f\_bijectivity}). Por tanto, podemos forzar que $f(0) = 1$ y $f(1) = 0$.

\begin{lstlisting}
  def permute_f {X Y : Type} [DecidableEq X]
    (f : X → Y) (a b : X) : X → Y := fun x ↦
      if x = a then f b
      else if x = b then f a
      else f x

  lemma permute_f_bijectivity {X Y : Type} [DecidableEq X]
      {f : X → Y} (a b : X) (h : f.Bijective) :
      (permute_f f a b).Bijective := by sorry
\end{lstlisting}

\begin{lstlisting}
  Nota: Todas estas demostraciones y la de hf son bastante largas y técnicas y no me parece que aporten mucho. Quizás se podrían dejar simplemente en el anexo.
\end{lstlisting}

Una vez demostrado \code{hf}, podemos definir $f$ mediante \code{Classical.choose} y empezar a trabajar con ella, aunque no la conozcamos explícitamente.

\begin{lstlisting}
  noncomputable def f : ℕ → Q := Classical.choose hf
\end{lstlisting}

Por ejemplo, podemos probar que tiene inversa.

\begin{lstlisting}
  lemma f_has_inverse :  ∃ g, Function.LeftInverse g f ∧
      Function.RightInverse g f := by
    rw [← Function.bijective_iff_has_inverse]
    exact f_prop.left
\end{lstlisting}

\subsubsection{Encontrar el sucesor y predecesor inmediato}

Ahora tenemos cada $p_k$ definido en Lean como $f(k)$ para cada $k \in \nat$. Para poder definir cada abierto $U_{p_k}$, necesitamos ser capaces de encontrar para cada conjunto $\{p_0, p_1, \dots, p_{n-1}\}$, el predecesor inmediato $p_r$ y el sucesor inmediato $p_s$ de $p_{n}$.

De nuevo, en Lean esto se codifica como funciones; queremos encontrar una función $r : \nat \to \nat$ que, para cada $n>1$, devuelva el predecesor inmediato de $f(n)$, de entre $\{f(k) ~|~ k < n\}$, y lo mismo para una función $s : \nat \to \nat$ que encuentre el sucesor inmediato. Sin embargo, la existencia de tales funciones no es trivial.

\begin{lemma}
  Sea $n > 1$. Entonces existe un $r_n < n$ de forma que $f(r_n) < f(n)$, y si $k < n$ es tal que $f(k) < f(n)$ entonces $f(k) \leq f(r_n)$.
\end{lemma}

\begin{lstlisting}
  lemma exists_r (n : ℕ) (hn : n > 1) : ∃ r ∈ Finset.range n,
    ((f r < f n) ∧
    (∀ m ∈ Finset.range n, f m < f n → f m ≤ f r)) := by sorry
\end{lstlisting}

\begin{proof}
  Sea $n>1$. Consideremos el conjunto
  $$
  R = \{m : \nat ~|~ m < n \land f(m) < f(n)\}
  $$
  \begin{lstlisting}
    let R : Finset ℕ := (Finset.range n).filter (fun m ↦ f m < f n) \end{lstlisting}
  $R$ es un conjunto finito no vacío, pues $1 \in R$
  \begin{lstlisting}
    have hR : R.Nonempty
    · use 1; sorry \end{lstlisting}

  Tomamos el conjunto $f(R)$, que también es un conjunto finito y no vacío, por serlo $R$, luego tiene máximo. Tomamos el argumento máximo de $f(R)$, $r_n = \arg \max \{f(m) ~|~ m \in R\}$, y veamos que satisface las condiciones que pedimos.

  \begin{lstlisting}
    let fR : Finset Q := R.image f
    have hfR : fR.Nonempty := (Finset.image_nonempty).mpr hR
    let fr := Finset.max' fR ((Finset.image_nonempty).mpr hR)
    obtain ⟨r, hr⟩ := Finset.mem_image.mp
      (by exact Finset.max'_mem fR hfR)
    use r \end{lstlisting}

  Se tiene que $r \in R$, luego $r < n$ y $f(r) < f(n)$. Sea entonces un $k < n$ con $f(k) < f(n)$. Por construcción, $k \in R$ y por tanto $f(k) \in f(R)$. Como $r$ es el argumento máximo de $f(R)$, $f(r)$ es el máximo de $f(R)$ y por tanto $f(k) \leq f(R)$, como queríamos.
\end{proof}

La demostración completa está en el anexo, así como el análogo para $s$.

Garantizada la existencia de estas funciones, podemos tomar ahora las funciones $r$ y $s$ y empezar a trabajar con ellas.

\begin{lstlisting}
  noncomputable def r : ℕ → ℕ := fun n ↦
    if h : n > 1 then Classical.choose (exists_r n h)
    else 1

  noncomputable def s : ℕ → ℕ := fun n ↦
    if h : n > 1 then Classical.choose (exists_s n h)
    else 0
\end{lstlisting}

Veamos las principales propiedades de estas dos funciones. En primer lugar tenemos las propiedades básicas de $r$ y $s$ que son simplemente por construcción.

\begin{lemma}
  Para cada $n > 1$, se tiene:
  $$
  \left\{
    \begin{array}{l}
      r(n) < n \\
      f(r(n)) < f (n) \\
      \forall m < n, \textnormal{ si } f(m) < f(n) \textnormal{ entonces } f(m) \leq f(r(n))
    \end{array}
  \right.
  $$
\end{lemma}

\begin{lstlisting}
  lemma r_prop (n : ℕ) (hn : n > 1) : (
      (r n ∈ Finset.range n) ∧ (f (r n) < f n) ∧
      (∀ m ∈ Finset.range n, f m < f n → f m ≤ f (r n))
    ) := by sorry
\end{lstlisting}

El resultado es simétrico para $s$ y recibe el nombre de \code{s\_prop}.

\begin{lemma}
  Sea $n > 1$. Entonces o bien $r(n) = 1$ o bien $r(n) > 1$. Es decir, no puede ser $r(n) = 0$. De forma análoga, no puede ser $s(n) = 1$, luego o bien $s(n) = 0$ o bien $s(n) = 1$.
\end{lemma}

\begin{lstlisting}
  lemma r_options (n : ℕ) (hn : n > 1) : r n = 1 ∨ r n > 1 := by sorry
  lemma s_options (n : ℕ) (hn : n > 1) : s n = 0 ∨ s n > 1 := by sorry
\end{lstlisting}

También se obtiene un resultado \code{rs\_options} que encapsula las cuatro combinaciones posibles de valores para $r$ y $s$, para facilitar el trabajo con ellas.

\begin{proof}
  Si fuera $r(n) = 0$, tendríamos $1 = f(0) = f(r(n)) < f(n)$, lo que es imposible pues $f$ toma valores en $[0, 1]$.
\end{proof}

\begin{lemma}
  Sea $n > 1$ y supongamos que $s(n) > 1$ y que $r(n) < s(n)$. Entonces $r(n) = r(s(n))$.

  De manera análoga, si $r(n) >1$ y $s(n) < r(n)$, entonces $s(n) = s(r(n))$.
\end{lemma}

\begin{lstlisting}
  lemma rn_eq_rsn (n : ℕ) (hn : n > 1) (hsn : s n > 1)
    (h : r n < s n) : r n = r (s n) := by
\end{lstlisting}

\begin{proof}
  Demostramos solo el primero y el segundo es parecido. Sea $n > 1$ y supongamos que $s(n) > 1$ y que $r(n) < s(n)$. Queremos ver que $r(n) = r(s(n))$. Por la inyectividad de $f$, podemos comprobar en su lugar que $ f (r (n)) = f (r (s (n)))$. Veamos que se cumplen ambas desigualdades.

  \begin{lstlisting}
    apply f_prop.left.left
    apply ge_antisymm \end{lstlisting}

  
\end{proof}

\subsubsection{Construcción de $G$}

La construcción de $G : \rat \to \mathcal{P}(X)$ es, como hemos explicado, una construcción por inducción. Para empezar, es más fácil construir $G : \nat \to \mathcal{P}(X)$, y después tomar $G \circ f^{-1} : Q \to \mathcal{P}(X)$ donde $f$ es la función que numera $Q$ que habíamos obtenido antes. Nos centraremos en esta función.

Lean admite definiciones inductivas de una manera muy natural. Un ejemplo muy utilizado es la sucesión de Fibonacci: definir $Fib(0) = 0$ y $Fib(1) = 1$, y a partir de ahí, $Fib(n) = Fib(n-1)+Fib(n-2)$ para cada $n > 1$. En Lean, escribimos

\begin{lstlisting}
  def Fib : ℕ → ℕ := fun n ↦
    if n = 0 then 0
    else if n = 1 then 1
    else Fib (n-1) + Fib (n-2)
\end{lstlisting}

En vista de esto, mi primer acercamiento a la construcción de $G$ fue el siguiente:

\begin{itemize}
  \item Para $n = 0$, definir $G(0) = C_2^c$.
  \item Para $n = 1$, tomar $G(1)$ el resultado de aplicar la caracterización de espacios normales al abierto $C_2^c$ y el cerrado $C_1$.
  \item Para $n >1$, tomar $G(n)$ el resultado de aplicar la caracterización de espacios normales al abierto $G(s(n))$ y el cerrado $\overline{G(r(n))}$.
\end{itemize}

\begin{lstlisting}
  def G {X : Type} [TopologicalSpace X]
    (hT : ∀ (U C : Set X), IsOpen U → IsClosed C → C ⊆ U →
      ∃ V, IsOpen V ∧ C ⊆ V ∧ closure V ⊆ U)
    (C1 C2 : Set X) (hC1 : IsClosed C1)
    (hC2 : IsOpen C2ᶜ) (hC1C2 : C1 ⊆ C2ᶜ)
    : ℕ → Set X := fun n ↦
      if n = 0 then C2ᶜ
      else if n = 1 then Classical.choose (hT C2ᶜ C1 hC2 hC1 hC1C2)
      else
        let U := g hT C1 C2 hC1 hC2 hC1C2 (s n)
        let C := closure (g hT C1 C2 hC1 hC2 hC1C2 (r n))
        Classical.choose (hT U C (by sorry) (by sorry) (by sorry))
\end{lstlisting}

Donde en los últimos \redcode{sorry} habría que demostrar que $G(s(n))$ es abierto, que $\overline{G(r(n))}$ es cerrado y que $\overline{G(r(n))} \subseteq G(s(n))$, para poder aplicar la caracterización de la normalidad (\code{hT}).

Todo esto nosotros \quotes{lo sabemos}: es la hipótesis de inducción. Pero a Lean todavía no le hemos dicho nada de eso. ¿Cómo podemos probar algo sobre un objeto que aún no hemos definido, porque necesitamos haberlo probado para poder definirlo?

Para evitar este problema, tomé una estrategia distinta. Primero, defino la noción de \textit{par normal} de la siguiente manera.

\begin{definition}
  Dados $U, C \subseteq X$, decimos que $(U, C)$ es un \textnormal{par normal} si $U$ es abierto, $C$ es cerrado y $C \subseteq U$. Es decir, si satisfacen las condiciones para poder aplicar la caracterización de espacios normales (\ref{caracterizacion-normal}).
\end{definition}

\begin{lstlisting}
  def normal_pair {X : Type} [TopologicalSpace X]
    : (Set X × Set X) → Prop := fun (U, C) ↦
    (IsOpen U ∧ IsClosed C ∧ C ⊆ U)
\end{lstlisting}

Ahora, defino una función \code{from\_normality} que toma dos conjuntos cualesquiera de $X$, y devuelve el resultado de aplicar la caracterización de espacios normales si forman un par normal, y el conjunto vacío en caso contrario.

\begin{lstlisting}
  noncomputable def from_normality {X : Type} [T : TopologicalSpace X]
    (hT : ∀ (U C : Set X), IsOpen U → IsClosed C → C ⊆ U → ∃ V, IsOpen V ∧ C ⊆ V ∧ closure V ⊆ U)
    : (Set X × Set X) → Set X := fun (U, C) ↦
      if h : normal_pair (U, C) = True then
        Classical.choose (hT U C
          (by simp at h; exact h.left)
          (by simp at h; exact h.right.left)
          (by simp at h; exact h.right.right)
        )
      else ∅
\end{lstlisting}

Ahora, al construir $G$, ya no me tengo que preocupar de si produce siempre conjuntos abiertos o no, porque lo puedo definir a partir de esta última función, que está definida para cualquieras dos abiertos. Una vez definida, puedo demostrar que cada conjunto obtenido es de hecho abierto.

\begin{lstlisting}
  def G {X : Type} [T : TopologicalSpace X]
    (hT : ∀ (U C : Set X), IsOpen U → IsClosed C → C ⊆ U →
      ∃ V, IsOpen V ∧ C ⊆ V ∧ closure V ⊆ U)
    (C1 C2 : Set X) : ℕ → Set X
      := fun n ↦
        if n = 0 then C2ᶜ
        else if n = 1 then from_normality hT (C2ᶜ, C1)
        else if n > 1 then
          let U := G hT C1 C2 (s n)
          let C := closure (G hT C1 C2 (r n))
          from_normality hT (U, C)
        else ∅
\end{lstlisting}

Sin embargo, a Lean esto sigue sin parecerle suficiente, y recibimos este error: \redcode{fail to show termination for G}. Esto es, estamos definiendo una función recursiva, $G$, pero no es evidente que las veces que estamos llamando a $G$ dentro de $G$ constituyan conjuntos ya definidos. Es decir, no es evidente que $r(n) < n$ y $s(n) < n$. Por tanto, al final de la definición tenemos que añadir una demostración de que sí es así.

\begin{lstlisting}
  ...
      else ∅

    decreasing_by
    · let s_prop := s_prop
      have aux : ∀ n > 1, s n < n
      · intro n hn
        specialize s_prop n hn
        simp at s_prop
        exact s_prop.left
      apply aux
      linarith
    · sorry -- analogo r
\end{lstlisting}

¡Ya tenemos nuestra función $G$! Aunque todavía queda mucho trabajo. Veamos que se cumplen las propiedades que queríamos de $G$.

\begin{lemma}
  Para cada $n \in \nat$, $G(n)$ es abierto en $X$.
\end{lemma}

\begin{lstlisting}
  lemma G_Prop1 {X : Type} [T : TopologicalSpace X]
    (hT : ∀ (U C : Set X), IsOpen U → IsClosed C → C ⊆ U →
      ∃ V, IsOpen V ∧ C ⊆ V ∧ closure V ⊆ U)
    (C1 C2 : Set X) (hC1 : IsClosed C1) (hC2 : IsOpen C2ᶜ)
    (hC1C2 : C1 ⊆ C2ᶜ)
    :
    ∀ n : ℕ, IsOpen (G hT C1 C2 n) := by sorry
\end{lstlisting}

\begin{proof}
  Notemos que la función \code{from\_normality} siempre produce abiertos, porque o bien es el resultado de aplicar \ref{caracterizacion-normal}, o bien es el vacío que es abierto (demostrado en \code{from\_normality\_open}).

  Sea $n \in \nat$. Tenemos que distinguir en tres casos, pues existen tres casos en la definición de $G$.

  (1) Si $n = 0$, entonces simplemente por hipótesis $G(0) = C_2^c$ es abierto.

  \begin{lstlisting}
    intro n
    cases' Nat.eq_zero_or_pos n with hn hn -- n = 0 ∨ n > 0
    · simp [hn, G] -- si n = 0
      exact { isOpen_compl := hC2 }
  \end{lstlisting}

  (2) Si $n = 1$, entonces estamos aplicando la función \code{from\_normality} a $C_2^c$ y $C_1$. Luego es abierto.

  \begin{lstlisting}
    cases' LE.le.eq_or_gt hn with hn hn -- n = 1 ∨ n > 1
    · simp [hn, G] -- n = 1
      exact from_normality_open hT C2ᶜ C1
  \end{lstlisting}

  (3) Ahora, si $n > 1$, entonces estamos aplicando la función \code{from\_normality} a $G(s(n))$ y $\overline{G(r(n))}$. Luego es abierto análogamente.
\end{proof}

Para el siguiente resultado vamos a utilizar inducción completa. Yo he definido mi propio principio de inducción completa, demostrado a partir de la inducción completa usual en Lean, por sencillez.

\begin{lstlisting}
  theorem my_stronger_induction (n : ℕ) (P Q : ℕ → Prop)
    (hn : P n) (h : ∀ n : ℕ, P n → ((∀ m < n, P m → Q m) → Q n)) :
    (Q n) := by sorry
\end{lstlisting}

\begin{lemma}
  Para cada $n > 1$, se tiene:
  $$
   \overline{G(r(n))} \subseteq G(n) \subseteq \overline{G(n)} \subseteq G(s(n))
  $$
\end{lemma}

\begin{lstlisting}
  lemma G_Prop2 {X : Type} [T : TopologicalSpace X]
    (hT : ∀ (U C : Set X), IsOpen U → IsClosed C → C ⊆ U →
      ∃ V, IsOpen V ∧ C ⊆ V ∧ closure V ⊆ U)
    (C1 C2 : Set X) (hC1 : IsClosed C1) (hC2 : IsOpen C2ᶜ)
    (hC1C2 : C1 ⊆ C2ᶜ)
    :
    ∀ n > 1, closure (G hT C1 C2 (r n)) ⊆ (G hT C1 C2 n)
      ∧ closure (G hT C1 C2 n) ⊆ (G hT C1 C2 (s n)) := by
\end{lstlisting}

\begin{proof}
  Procedemos por inducción completa sobre $n$.

  \begin{lstlisting}
    intro n hn
    let P : ℕ → Prop := fun m ↦ m > 1
    apply my_stronger_induction n P
    exact hn -- probar que n cumple P
    intro n hn hi \end{lstlisting}

  Tenemos la siguiente hipótesis de inducción: para cada $m < n$ con $m > 1$ se tiene que $\overline{G(r(m))} \subseteq G(m) \subseteq \overline{G(m)} \subseteq G(s(m))$.

  Queremos ver que entonces lo mismo se tiene para $n$.

  \begin{lstlisting}
  hi : ∀ m < n, 1 < m → closure (G hT C1 C2 (r m)) ⊆ G hT C1 C2 m ∧ closure (G hT C1 C2 m) ⊆ G hT C1 C2 (s m)

  ⊢ closure (G hT C1 C2 (r n)) ⊆ G hT C1 C2 n ∧ closure (G hT C1 C2 n) ⊆ G hT C1 C2 (s n) \end{lstlisting}

  Notemos que si $G(n)$ está obtenido mediante \ref{caracterizacion-normal} aplicado a $G(s(n))$ y $\overline{G(r(n))}$, lo anterior se reduce a comprobar que $(G(s(n)), \overline{G(r(n))})$ es un par normal (demostrado en \code{from\_normality\_prop2}).

  \begin{lstlisting}
    have normalpair : normal_pair (G hT C1 C2 (s n), closure (G hT C1 C2 (r n))) \end{lstlisting}

  Sabemos que $r(n)$ es o bien $r(n) = 1$ o bien $r(n) > 1$. No puede ser $r(n) = 0$ porque en ese caso sería $f(r(n)) = f(0) = 1 < f(n)$ (demostrado en \code{r\_options}). Análogamente, $s(n)$ es o bien $s(n) =0$ o bien $s(n) > 1$ (demostrado en \code{s\_options}).

  Tenemos que diferenciar cada una de las cuatro combinaciones posibles.

  Veamos por ejemplo el caso $r(n) = 1$, $s(n) > 1$. El resto son parecidos y se pueden consultar en el anexo.

  (1) Ver que $G(s(n))$ es abierto es sencillo; ya hemos visto que $G$ siempre devuelve abiertos.

  \begin{lstlisting}
    ...
      · exact G_Prop1 hT C1 C2 hC1 hC2 hC1C2 (s n) \end{lstlisting}

  (2) Ver que $\overline{G(r(n))}$ es cerrado es sencillo; ya hemos visto que la adherencia de cualquier conjunto es cerrada.

  \begin{lstlisting}
    ...
      · exact isClosed_closure \end{lstlisting}

  (3) Ahora, para ver que $\overline{G(r(n))} \subseteq G(s(n))$, necesitamos utilizar la hipótesis de inducción.

  Sabemos que $s(n) < n$. Luego podemos aplicar la hipótesis de inducción a $s(n)$, obteniendo que
  $$
  \overline{G(r(s(n)))} \subseteq G(s(n)) \subseteq \overline{G(s(n))} \subseteq G(s(s(n)))
  $$
  Pero utilizando que $r(n) = r(s(n))$, obtenemos $\overline{G(r(n))} \subseteq G(s(n))$.

  \begin{lstlisting}
    ...
      · have hsn := (s_prop n hn).left -- hsn : s n ∈ Finset.range n
        simp at hsn -- hsn : s n < n
        specialize hi (s n) hsn hs -- aplicar la H.I. a s(n)
        rw [rn_eq_rsn n hn hs (by linarith)] -- usar r(n) = r(s(n))
        exact hi.left \end{lstlisting}
\end{proof}

\subsubsection{La propiedad (\ref{eq:star})}



