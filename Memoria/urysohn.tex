

\section{El Lema de Urysohn}

[Introducción] \redcode{no se que poner}

\begin{theorem}[\textbf{Lema de Urysohn}]
  Sea $(X, \mathcal{T})$ un espacio topológico. $X$ es un espacio normal si y solo si para cada par de conjuntos cerrados disjuntos $C$ y $D$ en $X$, existe una función $f : X \to [0, 1]$ de manera que $f(C) = \{0\}$ y $f(D) = \{1\}$.
\end{theorem}

Para la fomalización en Lean, pediremos que los cerrados $C$ y $D$ sean no vacíos. Obviamente, si uno de los dos es vacío, basta tomar la función continua $f(x) \equiv 1$, pero podemos descartar estos casos triviales.

\begin{lstlisting}
  lemma Urysohn {X : Type} {Y : Set ℝ} (T : TopologicalSpace X)
      [T' : TopologicalSpace ℝ] (hT' : T' = UsualTopology)
      {R : TopologicalSpace Y} {hY : Y = Set.Icc 0 1}
      {hR : R = TopoSubspace T' Y} :
      NormalSpace X ↔
        ∀ C1 : Set X, ∀ C2 : Set X, C1 ≠ ∅ → C2 ≠ ∅ →
        IsClosed C1 → IsClosed C2 → Disjoint C1 C2 →
        ∃ f : X → Y, Continuous f ∧
          f '' C1 = ({⟨0, by simp [hY]⟩} : Set Y) ∧
          f '' C2 = ({⟨1, by simp [hY]⟩} : Set Y) := by sorry
\end{lstlisting}

Veamos primero la demostración del recíproco, que es más sencilla.

\begin{proof}

  Supongamos que cualquier par de cerrados disjuntos de $X$ se pueden separar mediante una función continua y veamos que entonces $X$ es un espacio normal. Sean $C_1$ y $C_2$ cerrados disjuntos en $X$.

  \begin{lstlisting}
    · intro h
      rw [normal_space_def]
      intro C1 C2 hC1 hC2 hinter \end{lstlisting}

  Como en la definición de normal no pedimos que los cerrados sean no vacíos, tenemos que diferenciar estos casos. Sin embargo, estos casos son triviales porque basta tomar el conjunto vacío para recubrir el vacío y $X$ para recubrir el otro conjunto. En Lean hay que ser rigurosos con este paso, pero aquí lo obviaremos por simplicidad.

  Supongamos entonces que $C_1$ y $C_2$ son no vacíos. Por hipótesis, existe una función continua $f : X \to [0, 1]$ de forma que $f(C_1) = \{0\}$ y $f(C_2) = \{1\}$.

  \begin{lstlisting}
    obtain ⟨f, hf, hfC1, hfC2⟩ := h C1 C2 hC1nempty hC2nempty hC1 hC2 hinter \end{lstlisting}

  Consideremos entonces los conjuntos $U_1 = f^{-1}([0, \frac{1}{2}))$ y $U_2 = f^{-1}((\frac{1}{2}, 1])$. Queremos ver que son los abiertos que necesitamos de la definición de normal, es decir, que son abiertos en $X$, que $C_i \subseteq U_i$ y que son disjuntos.

  \begin{lstlisting}
    use f ⁻¹' ({y | (y : ℝ) ∈ Set.Ico 0 (1 / 2)})
    use f ⁻¹' ({y | (y : ℝ) ∈ Set.Ioc (1 / 2) 1}) \end{lstlisting}

  Para ver que $U_1$ es abierto, utilizamos que $f$ es continua. Basta ver que $[0, \frac{1}{2})$ es abierto en $[0, 1]$. Pero ya vimos que los intervalos de la forma $[0, b)$ son abiertos en $[0, 1]$, así que basta aplicar esta propiedad. Análogo para $U_2$.

  \begin{lstlisting}
    · apply hf -- aplicar def. de f continua
      apply ico_open_in_Icc01 -- `[0, b)` es abierto en `[0, 1]`
      · exact hY -- estamos en [0, 1]
      · exact hR -- estamos en la top. relativa
      · norm_num -- 0 < 1/2 < 1 \end{lstlisting}

  Para ver que $C_1 \subseteq U_1$, basta ver que $f(C_1) \subseteq [0, \frac{1}{2})$. Que es trivial. Análogo para $U_2$.

  \begin{lstlisting}
    · rw [← Set.image_subset_iff, hfC1] -- `{0} ⊆ [0, 1/2)` ?
      simp \end{lstlisting}

  Para ver que son disjuntos, vemos que $[0, \frac{1}{2})$ y $(\frac{1}{2}, 1]$ son disjuntos. Obviamente lo son, pero para Lean es un poco más complicado, así que procedemos por reducción al absurdo para poder simplificar las expresiones. Finalmente llegamos a que no existe un $x$ con $x < 1/2$ y $x > 1/2$.

  \begin{lstlisting}
    · apply Disjoint.preimage
      by_contra c
      rw [Set.disjoint_iff_inter_eq_empty, ← ne_eq, ← Set.nonempty_iff_ne_empty] at c
      obtain ⟨x, hxu, hxv⟩ := c
      simp at hxu hxv
      linarith \end{lstlisting}
\end{proof}


La otra implicación es mucho más compleja, especialmente en su formalización en Lean, como veremos a continuación.


\subsection{Esquema de la demostración}

Para demostrar esta implicación, dados dos cerrados disjuntos no vacíos de $X$, queremos construir una función continua que los separe. La construcción de esta función ha constituido la parte más costosa de este trabajo.

Vamos a ver un esquema de los pasos a seguir para construir tal función.

Supongamos que $X$ es un espacio normal, y sean $C_1$ y $C_2$ dos cerrados disjuntos no vacíos de $X$.

Consideremos $U_1 = X$ abierto. Consideremos el cerrado $C_1$ el abierto $C_2^c$ y aplicamos la caracterización de espacios normales (\ref{caracterizacion-normal}), obteniendo otro abierto $U_0$ de manera que
$$
C_1 \subseteq U_0 \subseteq \overline{U_0} \subseteq C_2^c \subseteq U_1 = X
$$
Podemos hacer lo mismo para $\overline{U_0}$ cerrado y $C_2^c$ abierto, obteniendo $U_{\frac{1}{2}}$ de forma que
$$
C_1 \subseteq U_0 \subseteq \overline{U_0} \subseteq U_{\frac{1}{2}} \subseteq \overline{U_{\frac{1}{2}}} \subseteq C_2^c \subseteq U_1
$$
En general, vamos a construir una sucesión de abiertos sobre $\mathbb{Q}\cap[0, 1]$, $\{U_p | p \in \mathbb{Q}\cap[0, 1]\}$, de manera que
\begin{equation}
  \forall p , q \in \mathbb{Q}, p < q \implies \overline{U_p} \subseteq U_q \tag{$\star$} \label{eq:star}
\end{equation}

Una vez tenemos esta sucesión, que en Lean será una función $G : \mathbb{Q}\cap[0, 1] \to \mathcal{P}(X)$, definimos otra función $F$ sobre $X$ que a cada $x \in X$ le hace corresponder el conjunto
$$
F(x) = \{p \in \mathbb{Q} ~|~ x \in G(p)\}
$$
Por último, tomaremos la función $f : X \to [0, 1]$ definida por
$$
f(x) = \textnormal{inf}~F(x)
$$
Esta función será la que utilicemos. Tendremos que demostrar que efectivamente toma valores en $[0, 1]$, que es continua y que separa nuestros conjuntos cerrados.

Sin embargo, una vez construidas estas funciones, este último paso es relativamente sencillo. La principal dificultad a la hora de formalizar esta demostración ha sido el uso de la inducción para construir la función $G$ y demostrar sus propiedades.

Como se puede apreciar en las primeras iteraciones de la construcción de cada $U_q$, esta sucesión se construye por inducción. Para poder hacer inducción sobre los racionales, nos basamos en que son numerables, y, en particular, en que $\mathbb{Q}\cap[0, 1]$ lo es. Vamos a encontrar una función $f : \mathbb{N} \to \mathbb{Q} \cap [0, 1]$ biyectiva (invertible), de manera que $f(0) = 1$ y $f(1) = 0$. Esto nos servirá para construir cada $U_q$.

Después, para demostrar que efectivamente se cumple la condición \ref{eq:star}, necesitaremos utilizar inducción sobre dos variables. Para ello, vamos a demostrar que el orden lexicográfico de $(\mathbb{N} \times \mathbb{N})$, definido por $(n, m) < (n', m') \iff n<n' \lor (n=n' \land m<m')$, es una relación bien fundada, y por tanto admite inducción sobre pares de naturales.

\begin{lstlisting}
  Nota: no se como poner que lo más difícil ha sido que no se me han ocurrido estas cosas a la primera jajajaja
\end{lstlisting}

\subsection{Numerar los racionales}

Los racionales son denumerables, es decir existe una biyección entre $\nat$ y $\rat$.

En particular, existe una función $f : \nat \to Q$ donde $Q = \rat \cap [0, 1]$, de forma que

\begin{enumerate}
  \item $f$ es biyectiva
  \item $f(0) = 1$
  \item $f(1) = 0$
\end{enumerate}

A partir de ahora llamarmeos $f$ a esta función.

Demostración:

Demostración en Lean: [poner solo las partes importantes?]

\subsection{Una parte}

Sea $n \in \nat$ con $n > 1$. Como $$P_n = \{1, 2, \dots, n-1\}$$ es un conjunto finito y $f$ es sobreyectiva, existen $r$ y $s$ en $P_n$ (es decir, $r < n$ y $s < n$) tales que

\begin{equation} \label{cond_rs}
  f(r) < f(n) < f(s)
\end{equation}

y, además, estas son las mejores elecciones, es decir

\begin{enumerate}
  \item Si $m < n$ es tal que $f(m) < f(n)$, entonces $f(m) \leq f(r)$
  \item Si $m < n$ es tal que $f(n) < f(m)$, entonces $f(s) \leq f(m)$
\end{enumerate}

Además, como $f$ es inyectiva, estas elecciones son únicas.

Para cada $n > 1$, consideramos las funciones $r : \nat \to \nat$ y $s : \nat \to \nat$ que nos dan precisamente estos números $r$ y $n$.

--

Supongamos que hemos encontrado una función $G : \nat \to \mathcal{P}(X)$ que satisface las siguientes propiedades

\begin{enumerate}
  \item $G(0) = X \backslash C_2$
  \item $G(1) = V$ tal que $C_1 \subseteq V \subseteq \overline{V} \subseteq X \backslash C_2$
  \item Para cada $n \in \nat $, $G(n)$ es abierto en $X$
  \item Para cada $n \in \nat$, si $n>1$ entonces \begin{equation} \label{cond_Grs} \overline{G(r(n))} \subseteq G(n) \subseteq \overline{G(n)} \subseteq G(s(n)) \end{equation}
\end{enumerate}

Queremos probar que, en este caso, $G$ también cumple la siguiente propiedad:

\begin{equation} \label{cond_G}
  \forall n, m \in \nat, f(n) < f(m) \implies \overline{G(n)} \subseteq G(m)
\end{equation}