

\section{El Lema de Urysohn}

Introducción:

Definición: espacio normal

Lema (de Urysohn):

Esta demostración es relativamente complicada en papel, pero todavía lo es más cuando intentamos completarla en Lean. Por claridad, dividimos la demostración en varios pasos.

La demostración en Lean empieza utilizando la táctica \bluecode{constructor}, que nos divide el objetivo en dos, uno para cada implicación. Nos centraremos primero en la segunda, que es la más sencilla.

\subsection{El recíproco}





Queremos 

Demostración


\subsection{Numerar los racionales}

Los racionales son denumerables, es decir existe una biyección entre $\nat$ y $\rat$.

En particular, existe una función $f : \nat \to Q$ donde $Q = \rat \cap [0, 1]$, de forma que

\begin{enumerate}
  \item $f$ es biyectiva
  \item $f(0) = 1$
  \item $f(1) = 0$
\end{enumerate}

A partir de ahora llamarmeos $f$ a esta función.

Demostración:

Demostración en Lean: [poner solo las partes importantes?]

\subsection{Una parte}

Sea $n \in \nat$ con $n > 1$. Como $$P_n = \{1, 2, \dots, n-1\}$$ es un conjunto finito y $f$ es sobreyectiva, existen $r$ y $s$ en $P_n$ (es decir, $r < n$ y $s < n$) tales que

\begin{equation} \label{cond_rs}
  f(r) < f(n) < f(s)
\end{equation}

y, además, estas son las mejores elecciones, es decir

\begin{enumerate}
  \item Si $m < n$ es tal que $f(m) < f(n)$, entonces $f(m) \leq f(r)$
  \item Si $m < n$ es tal que $f(n) < f(m)$, entonces $f(s) \leq f(m)$
\end{enumerate}

Además, como $f$ es inyectiva, estas elecciones son únicas.

Para cada $n > 1$, consideramos las funciones $r : \nat \to \nat$ y $s : \nat \to \nat$ que nos dan precisamente estos números $r$ y $n$.

--

Supongamos que hemos encontrado una función $G : \nat \to \mathcal{P}(X)$ que satisface las siguientes propiedades

\begin{enumerate}
  \item $G(0) = X \backslash C_2$
  \item $G(1) = V$ tal que $C_1 \subseteq V \subseteq \overline{V} \subseteq X \backslash C_2$
  \item Para cada $n \in \nat $, $G(n)$ es abierto en $X$
  \item Para cada $n \in \nat$, si $n>1$ entonces \begin{equation} \label{cond_Grs} \overline{G(r(n))} \subseteq G(n) \subseteq \overline{G(n)} \subseteq G(s(n)) \end{equation}
\end{enumerate}

Queremos probar que, en este caso, $G$ también cumple la siguiente propiedad:

\begin{equation} \label{cond_G}
  \forall n, m \in \nat, f(n) < f(m) \implies \overline{G(n)} \subseteq G(m)
\end{equation}