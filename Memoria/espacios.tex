\section{Espacios topológicos en Lean}

Explicar algunos ejemplos de definiciones y demostraciones, no a modo de explicación completa de los prerrequisitos de topología sino para tener un primer acercamiento sencillo a la topología en Lean.

\subsection{Espacios topológicos}

\subsubsection{Conjuntos abiertos}

\subsubsection{Conjuntos cerrados}

\subsection{Bases}

\subsection{Topología relativa}

\subsection{Continuidad}

\subsection{Separación}

\subsubsection{Espacios de Hausdorff}

\subsubsection{Espacios normales}

Introducción : ¿por qué son interesantes?

\begin{definition}
  Sea $X$ un espacio topológico. $X$ es \emph{normal} si para cada par de cerrados disjuntos $C, D \subseteq X$ existen abiertos disjuntos $U$ y $V$ en $X$ tales  que separan $C$ y $D$, es decir, $C \subseteq U$ y $D \subseteq V$ \textnormal{(véase \cite[p. 99]{willard2012general})}.
\end{definition}

En Lean, escribimos esta definición de la siguiente forma.

\begin{lstlisting}
  def NormalTopoSpace {X : Type} (T : TopologicalSpace X) : Prop :=
    ∀ C : Set X, ∀ D : Set X,
    IsClosed C → IsClosed D → C ∩ D = ∅ →
    ∃ U : Set X, ∃ V : Set X,
      IsOpen U ∧
      IsOpen V ∧
      C ⊆ U ∧
      D ⊆ V ∧
      U ∩ V = ∅
\end{lstlisting}

Ahora queremos dar una caracterización para este tipo de espacios, que nos facilitará el trabajo más adelante.

\begin{proposition}[Caracterización de la normalidad]
  Sea $X$ un espacio topológico. $X$ es normal si y sólo si para cada abierto $U$ y cada cerrado $C$ de $X$ tales que $C \subseteq U$, existe un abierto $V \subset X$ de forma que $C \subseteq V \subseteq \overline{V} \subseteq U$.
\end{proposition}

En Lean, escribimos:

\begin{lstlisting}
  lemma characterization_of_normal {X : Type}
    (T : TopologicalSpace X) :
    NormalTopoSpace T ↔
      ∀ U : Set X, ∀ C : Set X, IsOpen U → IsClosed C → C ⊆ U →
      ∃ V : Set X, IsOpen V ∧ C ⊆ V ∧ (Closure V) ⊆ U := by sorry
\end{lstlisting}

\begin{proof}
  Veamos primero una implicación, y luego la otra (utilizamos \bluecode{constructor}).

  ($\implies$) Supongamos que $X$ es un espacio normal (\code{hT}) y sean $U$ un abierto (\code{hU}) y $C$ un cerrado (\code{hC}) tales que $C \subseteq U$ (\code{hCU}).

\begin{lstlisting}
    intro hT U C hU hC hCU
\end{lstlisting}
  
  Puesto que $X$ es normal, por la definición, para $C$ y $U^c$ cerrados en $X$ obtenemos $V_1$ y $V_2$ abiertos (\code{V1\_open}, \code{V2\_open}) disjuntos (\code{hV}) tales que $C \subseteq V_1$ (\code{hCV}) y $U^c \subseteq V_2$ (\code{hUV}).

\begin{lstlisting}
  obtain ⟨V1, V2, V1_open, V2_open, hCV, hUV, hV⟩ :=
    hT C Uᶜ
    hC
    (by exact isClosed_compl_iff.mpr hU)
    (by rw [ABdisjoint_iff_AsubsBc, compl_compl]; exact hCU)
\end{lstlisting}

  Por supuesto, en Lean tenemos que especificar por qué $U^c$ es cerrado y por qué $U^c \subseteq V_2$. Ahora tenemos una hipótesis de la forma

\begin{lstlisting}
  h : IsOpen V1 ∧ IsOpen V2 ∧ C ⊆ V1 ∧ Uᶜ ⊆ V2 ∧ V1 ∩ V2 = ∅
\end{lstlisting}
  
  Tomamos como $V$ el $V_1$ obtenido de esta forma,

\begin{lstlisting}
    use V1
\end{lstlisting}

  Queremos ver que satisface las condiciones que le pedimos:

\begin{lstlisting}
  ⊢ IsOpen V1 ∧ C ⊆ V1 ∧ Closure V1 ⊆ U
\end{lstlisting}
  
  Cómo tiene que cumplir tres condiciones, tendremos que utilizar \bluecode{constructor} varias veces. En primer lugar, $V_1$ es abierto por construcción. Además, $C \subseteq V_1$ también por construcción.

\begin{lstlisting}
    constructor
    · exact V1_open
    constructor
    · exact hCV
\end{lstlisting}

  Ahora queda demostrar que $\overline{V_1} \subseteq U$. Por un lado, tenemos que $V_1$ y $V_2$ son disjuntos, luego, en particular, como $V_2$ es abierto, se tiene que $\overline{V_1}$ y $V_2$ son disjuntos.

\begin{lstlisting}
    · apply disjointU_V_then_disjointClosureU_V V2_open at hV
      apply Set.disjoint_iff_inter_eq_empty.mpr at hV -- usamos la propiedad Disjoint de Lean
\end{lstlisting}

  Por otro lado, tenemos que $\overline{V_1} \subseteq U$ $\iff$ $V_1$ y $U^c$ son disjuntos. Basta ver que lo son utilizando lo anterior, sabiendo que $U^c \subseteq V_2$.

\begin{lstlisting}
      apply Set.disjoint_compl_right_iff_subset.mp
      exact Set.disjoint_of_subset_right hUV hV
\end{lstlisting}

  ($\Longleftarrow$) Procedemos de manera similar. Sean $C_1$, $C_2$ cerrados (\code{C1\_closed}, \code{C2\_closed}) disjuntos (\code{hC}). Podemos aplicar la hipótesis (\code{h}) al abierto $C_1^c$ y al cerrado $C_2$ para obtener obtener un abierto $V$ (\code{V\_open}) de manera que $C_2 \subseteq V \subseteq \overline{V} \subseteq C_1^c$ (\code{hV}).

\begin{lstlisting}
    intro h C1 C2 C1_closed C2_closed hC
    obtain ⟨V, V_open, hV⟩ :=
      h C1ᶜ C2
      (by exact IsClosed.isOpen_compl)
      C2_closed
      (by rw [← ABdisjoint_iff_AsubsBc, Set.inter_comm C2 C1]; exact hC)
\end{lstlisting}

  Ahora tomamos los abiertos $U_1 = \overline{V}^c$ y $U_2 = V$. Queremos ver que cumplen la condición de normalidad para $C_1$ y $C_2$.
  
\begin{lstlisting}
  IsOpen (Closure V)ᶜ ∧ IsOpen V ∧ C1 ⊆ (Closure V)ᶜ ∧ C2 ⊆ V ∧ (Closure V)ᶜ ∩ V = ∅
\end{lstlisting}
  
  En efecto, ambos son abiertos ($\overline{V}^c$ por ser el complementario de una clausura y $V$ por construcción).

\begin{lstlisting}
    constructor
    · simp
      exact closure_is_closed V
    constructor
    · exact V_open
\end{lstlisting}

  Además, $C_1 \subseteq \overline{V}^c$ es equivalente a $\overline{V} \subseteq C_1^c$, que es cierto por construcción de $V$, igual que $C_2 \subseteq V$.
  
\begin{lstlisting}
    constructor
    · apply Set.subset_compl_comm.mp
      exact hV.right
    constructor
    · exact hV.left
\end{lstlisting}

  Por último, se tiene

  $$
  \overline{V}^c \cap V = \emptyset \iff V \cap \overline{V}^c = \emptyset \iff
  V \subseteq \overline{V}^{cc} \iff V \subseteq \overline{V},
  $$

  que es cierto por las propiedades de la adherencia.

\begin{lstlisting}
    · rw [Set.inter_comm]
      rw [ABdisjoint_iff_AsubsBc]
      simp
      exact set_inside_closure V
\end{lstlisting}


\end{proof}